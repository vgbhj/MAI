\documentclass[11pt,reqno]{amsart}
%        Обязательные пакеты
\usepackage[T2A]{fontenc}
\usepackage[russian,english]{babel}
\usepackage[utf8]{inputenc}
\usepackage{izvuz7}
\usepackage{amssymb,amsmath,amsthm,amsfonts}
\usepackage{graphicx}

%        Дополнительные пакеты
% \usepackage[mathscr]{eucal} % = \mathscr{}, "выпрямленный" рукописный
\usepackage{multicol}
%        Операторы, например
% \DeclareMathOperator{\const}{const}
% \DeclareMathOperator*{ }{ }  %% оператор с пределами
% \newcommand{\U}{\mathfrak{U}}
% ^^^^^^^^^^^ такие конструкции желательно использовать как можно меньше!


% ВНИМАНИЕ: Из окончательного варианта файла убрать все неиспользуемые окружения!

%        Курсивные окружения
\theoremstyle{plain}
\newtheorem{theorem}{Теорема}
\newtheorem{lemma}{Лемма}
\newtheorem{proposition}{Предложение}
\newtheorem{axiom}{Аксиома}
\newtheorem{conjecture}{Гипотеза}
%% и другие по аналогии при необходимости

%        Подчиненные счетчики в окружениях
%\newtheorem{lemmac}[theorem]{Лемма}
%\newtheorem{propositionc}[theorem]{Предложение}
%\newtheorem{conjecturec}[theorem]{Гипотеза}
% Теорема 1, предложение 2, гипотеза 3...

%        Двойная нумерация окружений
\newtheorem{theorems}{Теорема}[section]
\newtheorem{lemmas}{Лемма}[section]
\newtheorem{propositions}{Предложение}[section]
% Теорема 1.1, лемма 1.1, предложение 1.1 и т.п.

%        Ненумерованные окружения
\newtheorem*{theoremnn}{Теорема}
% Просто теорема (без номера)

%        Окружения с прямым шрифтом
\theoremstyle{definition}
\newtheorem{definition}{Определение}
\newtheorem{corollary}{Следствие}
\newtheorem{remark}{Замечание}
\newtheorem{example}{Пример}

%        Аналогично ненумерованные с прямым шрифтом

% \numberwithin{equation}{section}  %%  для нумерации уравнений типа (1.1)

%        Заголовок статьи.
\tit{\MakeUppercase{НЕОБХОДИМЫЕ И ДОСТАТОЧНЫЕ УСЛОВИЯ ЭКСТРЕМУМА ДЛЯ ПОЛИНОМОВ И СТЕПЕННЫХ РЯДОВ В СЛУЧАЕ ДВУХ ПЕРЕМЕННЫХ}}
% \shorttit{\MakeUppercase{Сокращенный заголовок статьи для нечетных страниц}} % верхний колонтитул на нечетных страницах
 
%        Автор(ы), инициалы, фамилия
\author{\MakeUppercase{В.Н.\,НЕФЁДОВ}}
 

\god{202\_}         %% заполняется редакцией
\nomer{\textnumero\,\_}     %% заполняется редакцией

\setcounter{page}{1}
\pp{\pageref{firstpage}--\pageref{lastpage}}

%Пользовательские макросы
\renewcommand*{\proof}{\textbf{Доказательство.}}
\newtheorem{statement}{Утверждение}

\begin{document}
\Russian
%% Аннотация на русском

найдутся два числа, при одном из которых $\Bigl\{\left({\tilde{c}}_0, {\tilde{d}}_0\right),\left({-\tilde{c}}_0, {-\tilde{d}}_0\right)\Bigl\} = C_3$, а при другом $\Bigl\{\left({\tilde{c}}_0, {\tilde{d}}_0\right),\left({-\tilde{c}}_0, {-\tilde{d}}_0\right)\Bigl\} = C_4$. Рассмотрим случай 1) (остальные рассматриваются аналогично).

Действительно, в случае 1) достаточно положить $\tau = c^{-A_2}_0 = c^{e_1}_0 = |c_0|^{e_1}$ (при этом $|c_0| = \tau^{-1/A_2}$, $\tau=|c_0|^{-A_2}=(c_0)^{-A_2}$). Тогда, учитывая то, что в силу $u_0 = (c_0)^{e_1}(d_0)^{e_2} = |c_0|^{e_1}(d_0)^{e_2}$, выполняется $(u_0)^{\frac{1}{A_1}} = (u_0)^\frac{1}{e_2} = d_0|c_0|^{\frac{e_1}{e_2}} = d_0|c_0|^{-\frac{A2}{A1}}$, получаем, что для $\tilde{c_0} = \pm c_0 \tau^{1/A_2}, \tilde{d_0} = d_0 \tau^{1/A_1}$ справедливо: $|\tilde{c_0}| = |c_0|\tau^{1/A_2} = 1, \tilde{d_0} = d_0\tau^{1/A_1} = d_0|c_0|^{-\frac{A_2}{A_1}} = (u_0)^{\frac{1}{A1}} = (u_0)^{\frac{1}{e_2}}$.

Вернемся к задаче проверки условия (У4)$_{A,u_0}$ для некоторых $A \in \mathbf{A}_p, u_0 \in U(A)$ Покажем, что эту проверку достаточно проводить для конечного числа наборов $c_0 \ne 0$, $d_0 \ne 0$ таких, что $u_0 = (c_0)^{e_1}(d_0)^(e_2) \in U(A)$.

Воспользуемся тем, что для вектора $A \in \mathbf{A}_p$ выполняется один из перечисленных
ранее случаев относительно четности или нечетности его компонент. Пусть, например,
выполняется случай 1) (когда $e_1 = -A_2$ является четным, а $e_2 = A_1$ нечетным). Остальные
случаи рассматриваются аналогично. Предположим, что для некоторого $u_0 \in U(A)$ выполняется условие (У4)$_{A,u_0}$. Тогда существует $c_0 \ne 0$, $d_0 \ne 0$ такие, что для многочленов вида(20), выполняются условия: (21), $u_0 = (c_0)^{e_1}, (d_0)^{e_2}$, и справедливо (23). 
Рассмотрим подстановку: $t = \left(\tau^{1/(\nu_0A_1A_2)}\right)\tilde{t}, \tau > 0$. Тогда, учитывая(21), получаем

$$p(x(t),y(t)) = p\bigg(x\bigg(\bigg( \tau^{\frac{1}{\nu_0A_1A_2}}\bigg)\bigg)\bigg), y\bigg(\bigg( \tau^{\frac{1}{\nu_0A_1A_2}}\bigg)\bigg) = \tilde{g}_0\tilde{t}^{\sigma}, \text{где } \tilde{g}_0 = g_0\tau^{\frac{\sigma}{\nu_0A_1A_2}} < 0,$$

При этом

$$x(t) = x\bigg( \tau^{\frac{1}{\nu_0A_1A_2}}\tilde{t}\bigg) = c_0\bigg( \tau^{\frac{\nu_1}{\nu_0A_1A_2}}\bigg)\tilde{t}^{\nu_1} + c_1\bigg( \tau^{\frac{\nu_1+1}{\nu_0A_1A_2}}\bigg)\tilde{t}^{\nu_1+1} + o(\tilde{t}^{v_1+1})$$
$$y(t) = y\bigg( \tau^{\frac{1}{\nu_0A_1A_2}}\tilde{t}\bigg) = c_0\bigg( \tau^{\frac{\nu_2}{\nu_0A_1A_2}}\bigg)\tilde{t}^{\nu_1} + c_1\bigg( \tau^{\frac{\nu_2+1}{\nu_0A_1A_2}}\bigg)\tilde{t}^{\nu_2+1} + o(\tilde{t}^{v_1+1})$$

Заметим, что $\tilde{c}_0 = c_0 \left(\tau^{1/(\nu_0A_1A_2)}\right) = c_0\tau^{1/A_2} \text{, }\tilde{d}_0 = d_0 \left(\tau^{1/(\nu_0A_1A_2)}\right) = d_0\tau^{1/A_1}$, а следовательно, при выборе $\tau = c_0^{-A_2} = c^{e_1}_0 = |c_0|^{e_1}$ (как было показано при рассмотрении случая 1)) выполняется: $\left( \tilde{c}_0,\tilde(d)_0\right) \in C_1 = \Bigl\{\left(1, (u_0)^{1/e_2}\right),\left(-1, (u_0)^{1/e_2}\right)\Bigl\}$. т.е. $\tilde{d}_0 = (u_0)^{\frac{1}{e_2}}$ выбирается однозначно, а $\tilde{c}_0 \in \{1, -1\}$. Таким образом, мы показали, что при проверке для некоторых $A \in \mathbf{A}_p, u_0 \in U(A)$ условия (У4)$_{A,u_0}$ в случае 1) на многочлены вида (20), удовлетворяющие
(21), можно наложить дополнительное условие $(c_0, d_0) \in C_1$, которое из бесконечного числа
случаев выполнения $u_0 = (c_0)^{e_1}(d_0)^{e_2}$ позволяет ограничиться рассмотрением лишь двух из
них. Аналогичная ситуация имеет место и для остальных двух случаев. Таким образом
вместо условия (У4)$_{A,u_0}$ можно рассматривать эквивалентное ему условие

($\tilde{\text{У}}$4)$_{A,u_0}$ Существуют $c_0 \ne 0, d_0 \ne 0$ такие, что для многочленов вида (20), выполняются условия: (21), (23) и при этом в случае 1) $(c_0, d_0) \in C_1$, в случае 2) $(c_0, d_0, \in C_2)$, а в случае 3) $(c_0, d_0) \in C_3$ (или $(c_0, d_0) \in C_4$; выбираем любое из этих условий).

Рассмотрим теперь вопрос о выборе для многочленов вида (20), вектора $(\nu_1, \nu_2) \in \mathbb{N}^2$ при проверке условаия ($\text{У}$4)$_{A,u_0}$ или ($\tilde{\text{У}}$4)$_{A,u_0}$.  Одним из ограничений является условие (23) при выполнении которого должно найтись число $\nu \in \mathbb{N}$ такое, что $(\nu_1, \nu_2) = \nu A$. В связи с этим возникает вопрос: можно ли при этом обойтись случаем $\nu = 1$ ? Следующий пример показывает, что при применении метода подстановки многочленов с
неопределенными коэффициентами не всегда можно ограничиться значением $\nu = 1$.

\begin{example}
{
Пусть $p(x, y) = (x-y)^6 - (x-y)^2x^5 + x^8$. Тогда для $A = (A_1, A_2) = (1,1) p(x,y)=\phi^A_1(x,y)+\phi^A_2(x,y)+\phi^A_3(x,y)$, где

$\phi^A_1(x,y) = (x-y)^6 = x^6g^A_1(u),g^A_1(u)=(1-u)^6, B_1 = 6,$

$\phi^A_2(x,y) = (x-y)^2x^5 = x^7g^A_2(u),g^A_2(u)=(1-u)^2, B_2 = 7,$

$\phi^A_3(x,y) = x^8 = x^8g^A_3(u),g^A_3(u)=1, B_1 = 8.$
Характеристический многочлен $g^A_1(u)=(1-u)^6$ имеет единственный действительный
корень $u_0=1$. Заметим, что $A=(A_1, A_2)=(1,1) \in \mathbf{A}_p$, и при проверке условия ($\tilde{\text{У}}$4)$_A$ в общем случае должны рассматривать многочлены $x(t), y(t)$ вида
\begin{equation}\label{nef:eq:24}
x(t) = c_0t^{\nu} + c_1t^{\nu+1}+o(t^{\nu+1}), y(t) = d_0t^{\nu} + d_1t^{\nu+1} + o(t^{\nu+1}) \text{, где } \nu \in \mathbb{N}\end{equation}
Здесь имеет место случай 3) (когда оба числа $e_1 = -A_2 = -1, e_2 = A_1 = 1$  являются нечетными), и в этом случае числа $c_0, d_0$ можно выбирать из множества $C_3 = {(1,1),(-1,-1)}$. В случае $\nu = 1$ при $c_0 = 1, d_0 = 1$ многочлены (23) имеют вид (для простоты обознпчаем $c_1 = c, d_1 = d$)
\begin{equation}\label{nef:eq:25}
x(t) = t + ct^2 + o(t^2),y(t) = t + dt^2 + o(t^2).
\end{equation}
Тогда $\phi^A_1(x(t),y(t)) = (x(t) - y(t))^6 = (c-d)^6t^12+o(t^12),$ \\
$\phi^A_2(x(t),y(t))=-(x(t)-y(t))^2 \left[x(t)\right] = -(c-d)^2t^9+o(t^9),\phi^A_3(x(t),y(t))=\left[x(t)\right]^8 = t^8 + o(t^8)$. \\
Таким образом, $p(x(t),y(t)) = (c-d)^6t^12 + o(t^12)-(c-d)^2t^9 + o(t^9)+t^8+o(t^8)$,\\
а следовательно, для любых многочленов вида (25) выполняется $p(x(t),y(t)) = t^8 + o(t^8)$.\\
Совершенно аналогично в случае выбора $c_0 = -1,d_9 = -1$, т.е. для многочленов вида $x(t) = -t + ct^2 + o(t^2), y(t)=-t+dt^2+o(t^2)$ также выполняется $p(x(t),y(t)) = t^8 + o(t^8)$.

Рассмотрим теперь многочлены вида (24) при $\nu = 2, c_0 = 1, d_0 = 1$ (снова для
простоты обозначаем $c_1 = c, d_1 = d$):
$$x(t) = t^2 + ct^3 + o(t^3), y(t)=t^2+dt^3+o(t^3).$$
Тогда $\phi^A_1(x(t),y(t)) = (x(t)-y(t))^6 = (c-d)^6t^{18}+o(t^{18}),\phi^A_2(x(t),y(t))=-(x(t)-y(t))^2x^5(t)=\\
=-(c-d)^2t^{16}+t^{16},\phi^A_3(x(t),y(t))=x^8(t)=t^{16}+o(t^{16}),p(x(t),y(t))=(c-d)^6t^{18}+o(t^{18})-\\
-(c-d)^2t^{16}+o(t^{16})+t^{16}+o(t^{16})$, а следовательно, например, при $c=2,d=0$ для многочленов $x(t)=t^2+2t^3$, $y(t)=t^2$ выполнется $p(x(t),y(t)) = 2^6t^{18}-2^4t^{16}+t^{16} = -15t^15+o(t^{16})$. Таким образом, рассмотрение многочленов вида (26) показало, что $0_{(2)}$ не является точкой локального минимума полинома $p(x,y)$, т.е.
рассмотрение только многочленов вида (25), соответствующих случаю $\nu = 1$, оказалось
недостаточным. Отметим, что при использовании алгоритма 1 мы не получим ответа на
вопрос, является ли $0_{(2)}$ точкой локального минимума полинома $p(x,y)$ из этого примера.

}
\end{example}
\begin{remark}{
Все используемые в этом разделе утверждения остаются в силе и для степенного ряда $p(x,y)$ $[6]$,  а следовательно могут быть применены к $p(x,y)$ и в этом
случае.
}
\end{remark}
\section{Случай, когда $p(x,y)$ является суммой двух квазиоднородных форм.}
\label{nef:sec:3}
Рассмотрим теперь один частный случай, когда полином $p(x,y)$ удовлетворяющий условиям утверждения 8, является суммой двух квазиоднородных форм. Покажем, что в этом случае с помощью простых вычислительных процедур можно однозначно ответить на вопрос, является ли $0_{(2)}$ точкой локального минимума $p(x,y)$.

Нам потребуются следующие утверждения
\begin{statement}\label{nef:stt:9}
Пусть для некоторого $A \in \mathbb{N}^2_0$ разложение (6), (7) полинома $p(x,y)$ состоит из двух A-квазиоднородных форм, т.е. имеет вид $p(x,y) = \phi^A_1(x,y)+\phi^A_2(x,y)$. Тогда не существует вектора $\overline{A} \in \mathbb{N}^2_0$ такого, что $\overline{A} \ne A$, и главная $\overline{A}$-квазиоднородная
форма $\phi^{\overline{A}(x,y)}$ полинома $p(x,y)$ содержит более двух членов (мономов).
\end{statement}
\noindent \begin{proof}
Предположим, что для некоторого $\overline{A}=(\overline{A_1},\overline{A_2}) \in \mathbb{N}^2_0$ главная $\overline{A}$- квазоидная форма $\phi^{\overline{A}}_1(x,y)$ полинома содержит по крайней мере три члена (монома). При этом по крайней мере два из них принадлежат одной из двух форм $\phi^{A}_1(x,y)$ или $\phi^{A}_2(x,y)$. Поскольку прямая на плоскости однозначно определяется любыми двумя
точками, находящимися на этой прямой, то принадлежность двух мономов любой из этих
форм означает, что $\overline{A} = A$ (поскольку $A,\overline{A}\in \mathbb{N}^2_0$), т.е. пришли к противоречию с $\overline{A} \ne A$.
\end{proof}

\begin{statement}\label{nef:stt:10}
Пусть $A = (A_1, A_2 \in \mathbb{N}^2_0)$, $u_0 \in 0$, l - четное натуральное число, $p(x,y), \overline{p}(x,y)$ - полиномы, $p(x,y) = (y^{A_1} - u_0x^{A_2})^l\overline{p}(x,y)$. Тогда $0_{(2)}$ является точкой локального минимума полинома $p(x,y)$ тогда и только тогда, когда она является точкой локального минимума полинома $\overline{p}(x,y)$.
\end{statement}
\noindent \begin{proof}
Действительно, пусть $0_{(2)}$ не является точкой локального минимума полинома $p(x,y)$. Тогда найдется последовательность точек $(x(n),y(n)), n = 1,2,\dots,$ таких, что $p(x(n),y(n)) = {\left[\big(y(n) \big)^{A_1} - u_0\big(x(n) \big)^{A_2}\right]}^l\overline{p}(x(n),y(n)) < 0, (x(n),y(n)) \to 0_{(2)}$ при $n \to \infty$.
Из этих условий, используя четность l, получаем:
$$\left[ y(n)\right]^{A_1} - u_0\left[x(n)\right]^(A_2) \ne 0, \overline{p}((x(n),y(n))) < 0, n=1,2,\dots,$$
($u_0 \ne 0$ - не требуется!) т.е. $0_{(2)}$ не является точкой локального минимума полинома $\overline{p}(x,y)$.

\textbf{В обратную сторону.} Пусть $0_{(2)}$ не является точкой локального минимума полинома $\overline{p}(x,y)$. Тогда найдется последовательность точек $(x(n),y(n)), n=1,2,\dots,$ таких, что $\overline{p}((x(n),y(n))) < 0, (x(n), y(n)) \to 0_{(2)}$ при $n \to \infty$. При этом в силу непрерывности $\overline{p}(x,y)$ можно считать, что $x(n) \ne 0, y(n) \ne 0, n = 1,2,\dots$. Заметим, что, если для некоторых $x_0 \ne 0, y_0 \ne 0, u_0 \ne 0$ выполняется $y^{A_1}_0 - u_0x^{A_2}_0 = 0$, то $\forall \nu \in (0,1) (\nu y_0)^{A_1} - u_0x_0^{A_2} \ne 0$.\\
Действительно, если $(\nu y_0)^{A_1} - u_0x_0^{A_2} \ne 0, \ne > 0$, то $(\nu y_0)^{A_1} = y^{A_1}_0$, откуда $\nu^{A_1} = 1$, что противоречит условию $\nu \in (0,1)$. Но тогда, используя непрерывность $\overline{p}(x,y)$, для любого номера $n = 1,2,\dots$ можно подобрать такое число $\nu_n \in (1-1/n, 1]$, такое, что 
$\overline{p}(x(n),\nu_n y(n)) < 0, \left(\nu_n y(n)\right)^{A_1} - u_0(x(n))^{A_2} \ne 0$ (если $\left(y(n) \right)^{A_1} - u_0\left(y(n) \right)^{A_2} \ne 0$, то полагаем $\nu_n = 1$). Таким образом, получаем:
$$p(x(n),\nu_n y(n)) = \left[ \big( \nu_n y(n)\big)^{A_1} - u_0\big( x(n)\big)^{A_2} \right]^l \overline{p}(x(n),\nu_n y(n)) < 0, n = 1,2,\dots,$$
и при этом $(x(n),\nu_n y(n)) \to 0_{(2)}$ при $n \to \infty$, т.е. $0_{(2)}$ не является точкой локального
минимума полинома $p(x,y)$.
\end{proof}

Приведем некоторые сведения относительно произвольной A-квазиоднородной формы $\phi^A_1(x,y)$, где $A=(A_1,A_2) \in \mathbb{N}^2_0$, вида (2)-(4). Пусть, как и ранее, $e = (-A_2, A_1), u = x^{e_1}y^{e_2} = x^{-A_2}y{A_1}$,
\begin{align*}
    &\phi^A_1(x,y) = \displaystyle\sum_{i=1}^{s} a_i x^{\alpha_i}y^{\beta_1} = x^{\alpha_i}y^{\beta_1}g^A_1(x^{-A_2}y^{A_1}) = x^{\alpha_i}y^{\beta_1}g^A_1(x^{e_1}y^{e_2}) =  x^{\alpha_i}y^{\beta_1}g^A_1(u),\\
    &r_1 = \deg g^A_1(u) = \nu_s = (\alpha_1 - \alpha_s) / A_2, \alpha_1 = \alpha_s + r_1 A_2, \alpha_1 - r_1 A_2 = \alpha_s
\end{align*}
($\deg g(u)$ - степень многочлена $g(u)$). Пусть, далее, $u_0$ - корень кратности $k \in \mathbb{N}$ многочлена $g^A_1(u),$ т.е. $g^A_1(u) = (u - u_0)^k\overline{g}^A_1$, где $\overline{g}^A_1(u)$ - многочлен, $\overline{g}^A_1(u_0) \ne 0, \overline{r}_1 = \deg \overline{g}^A_1(u) = r_1 - k$. Тогда
$$
\phi^A_1(x,y) = x^{\alpha_i}y^{\beta_1}g^A_1(x^{-A_2}y^{A_1}) = x^{\alpha_i}y^{\beta_1}g^A_1(x^{-A_2}y^{A_1} - u_0)^k\overline{g}^A_1(x^{-A_2}y^{A_1} ) =\\
$$
\begin{equation}\label{nef:eq:27}
=(y^{A_1} - u_0x{A_2})^k x^{\alpha_1-r_1A_2}y^{\beta_1}\left[ x^{\overline{r}_1 A_2} \overline{g}^A_1 (x^{-A_2} y^{A_1})\right] = (y^{A_1} - u_0x^{A_2})^k \overline{\phi}^A_1(x,y),
\end{equation}
где полином
\begin{equation}\label{nef:eq:28}
\overline{\phi}^A_1(x,y) = x^{\alpha_1-r_1A_2}y^{\beta_1}\left[ x^{\overline{r}_1A_2}\overline{g}^A_1(x^{-A_2}y^{A_1}) \right] = x^{\alpha_1-kA_2}y^{\beta_1}y^{\beta_1}\overline{g}^A_1(x^{-A_2}y^{A_1})
\end{equation}
имеет члены вида $\overline{a}_i x^{\overline{\alpha}_i}y^{\overline{\beta}_i} = \overline{a}_i x^{\alpha_1 - k A_2}y^{\beta_1}(x^{-A_2}y^{A_1})^{\overline{\nu}_i} = \overline{a}_i x^{\alpha_1 - (k+\overline{\nu}_i) A_2}y^{\beta_1 + A_1 \overline{\nu}_i}, \overline{\nu}_i \in \mathbb{N} \cup \{0\}, \overline{\nu}_i \leq \overline{r}_1,$ где $\overline{\alpha}_i = \alpha_1 - (k + \overline{\nu}_i)A_2 \geq \alpha_1 - (k + \overline{r}_1)A_2 = \alpha_1 - r_1A_2 = \alpha_s \geq 0, \overline{\beta}_i = \beta_1 + A_1 \overline{\nu}_i \geq \beta_1,$
\begin{align*}
    &B_1 = A_1\alpha_1 + A_2\beta_1 = A_1(\alpha_s + r_1A_2) + A_2\beta_1 = r_1A_1A_2 + \alpha_sA_1 + A_2\beta_1,\\
    &A_1\overline{\alpha}_i + A_2\overline{\beta}_1 = A_1\left[\alpha_1 - (k + \overline{\nu}_i)A_2\right] + A_2(\beta_1 + A_1\overline{\nu}_i) = A_1 \left[ \alpha_1 - kA_2 \right] + A_2\beta_1 = \\
    & = A_1\alpha_1 + A_2\beta_1 - kA_1A_2 = B_1 - kA_1A_2 = (r_1 - k)A_1A_2 + \alpha_sA_1+A_2\beta_1 \geq 0.
\end{align*}

Таким образом $\overline{\phi}^A_1(x,y)$ также является А-квазиоднородной полиномиальной формой, для членов $a_i x^{\overline{\alpha}_i} y^{\overline{\beta}_i}$ которой выполняется
$$
\overline{B}_1 = A_1 \overline{\alpha}_i + A_2 \overline{\beta}_i = B_1 - kA_1A_2 = (r_1 - k)A_1A_2 + \alpha_s A_1 + A_2 \beta _1 \geq 0,
$$
и при этом $\overline{B_1} = 0 \implies r_1 = k, \alpha_s = 0, \beta_1 = 0$. Если $\overline{B_1} = 0$, то равенство $r_1 = k$ означает, что многочлен $\overline{g}^A_1(u)$ имеет степень $r_1 - k = 0$, т.е. является константой $G \ne 0$ (поскольку $\overline{g}^A_1(u) \ne 0$), и тогда
$$
\overline{\phi}^A_1(x,y) = x^{\alpha_1 - k A_2}y^{\beta_1} \overline{g}^A_1(u)(x^{-A_2}y^{A_1}) = Gx^{\alpha_1 - k A_2}y^{\beta_1} = Gx^{\alpha_s + r_1A_2 - kA_2}y^{\beta_1} = Gx^0y^0 = G,
$$
т.е. А-квазиоднородная форма $\overline{\phi}^A_1(x,y)$ является этой же константой.

Пусть $p(x,y)$ - полином, для которого выполнены условия утверждения 8: $p(x,y) \ne 0$,$p(0,0) = 0$, $p'(0,0) = 0^{(2)}$ (т.е. $0^{(2)}$ является стационарной точкой), все главные квазиоднородные формы полинома $p(x,y)$ из групп 1 и 2 являются неотрицательными и невырожденными в слабом смысле, и при этом $\mathbf{A}_p \ne \varnothing$,$A=(A_1, A_2)=(1,1) \in \mathbf{A}_p$. Пусть, далее,
$$
p(x,y) = \phi^A_1(x,y)+\phi^A_2(x,y) = x^{\alpha_1}y^{\beta_1}g^A_1(u)+x^{\gamma_1}y^{\eta_1}g^A_2(u),
$$
где $\phi^A_1(x,y)$ удовлетворяет условиям (2)-(4), а $\phi^A_2(x,y)$ - условиям (8)-(10). Как это следует из утверждения 9, в этом случае $\mathbf{A}_p = \{A\}$.

Пусть $u_0 \in U_p(A)$, и при этом кратность корня $u_0$ в многочлене $g_1(u)$ равна четному натуральному числу $k \in 2\mathbb{N}$ (см. утверждение 6.1), а кратность корня $u_0$ в многочлене $g_2(u)$ равна $l \in \mathbb{N} \cup {0}$. Тогда $g^A_1(u) = (u - u_0)^k \overline{g}^A_1(u), g^A_2(u) = (u - u_0)^l\overline{g}^A_2(u)$, где $\overline{g}^A_1(u_0) > 0, \overline{g}^A_2(u_0) \ne 0$. Пусть $r_1 = \deg g^A_1(u), \overline{r}_1 = \deg \overline{g}^A_1(u), r_2 = \deg g^A_2(u), \overline{r}_2 = \deg \overline{g}^A_2(u)$.\\
Возмодны следующие 4 случая.
\begin{enumerate}
    \item[1)] Пусть $k \geq 2$ четно и l = 0, т.е. $g^A_2(u_0) \ne 0$.
    \item[2)] Пусть $l < k$ l четно.
    \item[3)] Пусть $k \leq l$.
    \item[4)] Пусть $l < k$ l нечетно.
\end{enumerate}

Используя утверждение 10, нетрудно от полинома $p(x,y)$ перейти к полиному $\tilde{p}(x,y) = \tilde{\phi}^A_1(x,y)+\tilde{\phi}^A_2(x,y)$, также являющемуся суммой двух А-квазиоднородных форм, такому, что $U_{\tilde{p}}(A) \subseteq U_p (A)$, полином $\tilde{\phi}^A_1(x,y)$ является неотрицательным  и для
нового полинома $\tilde{p}(x,y)$ для любого $u_0 \in U_{\tilde{p}}(A)$ будут выполняться только случаи 1), 4) и при этом в случае 4) $l = 1$. Если при этом выполняется только случай 1), то мы находимся в области применимости алгоритма 1, используя который, получим однозначный ответ, является ли $0_{(2)}$ точкой локального минимума полинома $\tilde{p}(x,y)$ (а тем самым и $p(x,y)$).



\label{lastpage}  %% Не удалять!!!

\end{document}