%% Пожалуйста, прочитайте внимательно рекомендации к оформлению статьи,
%  содержащиеся в данном коммментарии

%% По возможности использовать стандартные окружения, определенные ниже
%
%% Закомментировать неиспользуемые окружения!
%
%% По возможности не использовать свои макросы и команды
%
%% Использовать разбиение статьи на разделы и параграфы командами \section{...}
% (\section*{...})
%
%% Использовать стандартное окружение \begin{proof}...\end{proof} для доказательств
%
%% Использовать команду \linebreak в крайней необходимости (эта команда
%  используется редактором для выравнивания текста в итоговом варианте)
%
%% Нумеровать только те формулы, на которые есть ссылка
%
%% Ссылки на формулы делать через присвоенные формулам лейбл \label{...},
%  а не напрямую ([1]). Например: \begin{equation}\label{ivanov:1} y=x^2 \end{equation}
%
%% Ссылаться на литературу также через ссылки: статья~\cite{ivanov:1}
%
%% Ссылки оформлять индивидуально, например, \label{ivanov:1},
%  иначе они путаются со одноименными ссылками других авторов (\label{1}, \label{t1}).
%
%% Ссылаться на формулы командой \eqref вместо (\ref)  
%
%% При ссылках на формулы, цитировании литературы, указании страниц,
%  инициалы и фамилии авторов и др. использовать неразрывный пробел:
%  формула~\eqref{ivanov:1},...
%  И.\,И.~Иванов
%  c.~21
%  статья~\cite{ivanov:1}, статья~[21]
%
%% По возможности, указывать в комментарии в библиографии ссылки
%  на имеющиеся англоязычные переводы статей
%
%% По возможности не использовать \vspace, \vskip, \par и т.д. для промежутков в статье
%
%% Разделять включную формулу на логические (грамматические) части:
%  неправильно: $u(t,v),   0 \leq t <+ \infty,\, v \in \mathbb{R}^{q},\, u(t,v)\in \mathbb{R}^{p},$
%  правильно:   $u(t,v)$, $0 \leq t <+ \infty$, $v \in \mathbb{R}^{q}$, $u(t,v)\in \mathbb{R}^{p}$,
%
%  Пунктуацию (знаки точки, запятой и т.д.) выносить за границы формул: 
%  неправильно: $u(t,v),$ где $u$
%  правильно:   $u(t,v)$, где $u$
%
%% Обратно, не разрывать слитные части формул: $0<t\leq$ $\leq 1$, $f(x)=$ $=1+x$.
%  Если нужно, это сделает редактор, вы можете указать это при проверке гранок статьи.
%
%% Ниже содержится шаблон оформления статьи.
%  В самом конце приводится пример вставки иллюстраций и таблиц.
%  Для каждой иллюстраций всегда прикладывайте eps и jpg файлы (они должен быть разрешением не менее 150 DPI),
%  не используйте встроенные средства построения графики (picture, tikz, )
%
%% Каждая иллюстрация и каждая таблица должны быть в плавающем окружении (figure, tabular).
%  В абзаце, идущем перед иллюстрацией или таблицей бязательно должна быть активная ссылка на эту иллюстрацию или таблицу (см. Рис.~\ref{...})
%

\documentclass[11pt,reqno]{amsart}
%        Обязательные пакеты
\usepackage[T2A]{fontenc}
\usepackage[russian,english]{babel}
\usepackage[utf8]{inputenc}
\usepackage{izvuz7}
\usepackage{amssymb,amsmath,amsthm,amsfonts}
\usepackage{graphicx}

%        Дополнительные пакеты
% \usepackage[mathscr]{eucal} % = \mathscr{}, "выпрямленный" рукописный
\usepackage{multicol}
%        Операторы, например
% \DeclareMathOperator{\const}{const}
% \DeclareMathOperator*{ }{ }  %% оператор с пределами
% \newcommand{\U}{\mathfrak{U}}
% ^^^^^^^^^^^ такие конструкции желательно использовать как можно меньше!


% ВНИМАНИЕ: Из окончательного варианта файла убрать все неиспользуемые окружения!

%        Курсивные окружения
\theoremstyle{plain}
\newtheorem{theorem}{Теорема}
\newtheorem{lemma}{Лемма}
\newtheorem{proposition}{Предложение}
\newtheorem{axiom}{Аксиома}
\newtheorem{conjecture}{Гипотеза}
%% и другие по аналогии при необходимости

%        Подчиненные счетчики в окружениях
%\newtheorem{lemmac}[theorem]{Лемма}
%\newtheorem{propositionc}[theorem]{Предложение}
%\newtheorem{conjecturec}[theorem]{Гипотеза}
% Теорема 1, предложение 2, гипотеза 3...

%        Двойная нумерация окружений
\newtheorem{theorems}{Теорема}[section]
\newtheorem{lemmas}{Лемма}[section]
\newtheorem{propositions}{Предложение}[section]
% Теорема 1.1, лемма 1.1, предложение 1.1 и т.п.

%        Ненумерованные окружения
\newtheorem*{theoremnn}{Теорема}
% Просто теорема (без номера)

%        Окружения с прямым шрифтом
\theoremstyle{definition}
\newtheorem{definition}{Определение}
\newtheorem{corollary}{Следствие}
\newtheorem{remark}{Замечание}
\newtheorem{example}{Пример}

%        Аналогично ненумерованные с прямым шрифтом

% \numberwithin{equation}{section}  %%  для нумерации уравнений типа (1.1)

%        Заголовок статьи.
\tit{\MakeUppercase{НЕОБХОДИМЫЕ И ДОСТАТОЧНЫЕ УСЛОВИЯ ЭКСТРЕМУМА ДЛЯ ПОЛИНОМОВ И СТЕПЕННЫХ РЯДОВ В СЛУЧАЕ ДВУХ ПЕРЕМЕННЫХ}}
% \shorttit{\MakeUppercase{Сокращенный заголовок статьи для нечетных страниц}} % верхний колонтитул на нечетных страницах
 
%        Автор(ы), инициалы, фамилия
\author{\MakeUppercase{Н.В.\,НЕФЁДОВ}}
 

\god{202\_}         %% заполняется редакцией
\nomer{\textnumero\,\_}     %% заполняется редакцией

\setcounter{page}{1}
\pp{\pageref{firstpage}--\pageref{lastpage}}

% \dedic{Посвящается ...}  %% Посвящение (при необходимости)
% Можно использовать \begin{flushright} ... \protect\\ \end{flushright}
% если посвящение в 2-3 строки

\begin{document}
 %% Не удалять!!!
\Russian
%% Аннотация на русском
\begin{minipage}{0.49\textwidth}
\raggedright
Известия вузов. Математика\\
2018, №1, с. 75-80
\end{minipage}
\begin{minipage}{0.49\textwidth}
\raggedleft
tp://kpfu.ru/science/nauchnye-izdaniya/ivrm/\\
e-mail: izvuz.matem@kpfu.ru    
\end{minipage}
\\
\begin{center}
    \emph{Н.В. НЕФЁДОВ}
\end{center}
\begin{center}
    \textbf{НЕОБХОДИМЫЕ И ДОСТАТОЧНЫЕ УСЛОВИЯ ЭКСТРЕМУМА ДЛЯ ПОЛИНОМОВ И СТЕПЕННЫХ РЯДОВ В СЛУЧАЕ ДВУХ ПЕРЕМЕННЫХ}
\end{center}
\abstract{Настоящая работа является продолжением серии работ автора, в которых приводятся некоторые достаточно тонкие необходимые и достаточные условия локального экстремума (для определенности – минимума) в стационарной точке полинома или степенного ряда (а тем самым – аналитической функции). Известно, что для случая одной переменной необходимые и достаточные условия «смыкаются», т.е. их легко сформулировать как единое условие. Следующим по сложности является случай с двумя переменными, который и рассматривается в настоящей работе. В этом случае многие процедуры, к которым сводится проверка необходимых и достаточных условий, полученных в предыдущих работах автора, основываются на вычислении действительных корней многочлена от одной переменной, а также на решении некоторых других достаточно простых практически реализуемых задач. В настоящей работе удалось описать алгоритм, базирующийся на этих процедурах, предоставляющий возможность для решения исследуемой задачи даже в очень тонких случаях. Тем не менее, все же остаются случаи, когда этот алгоритм «не работает». Для таких случаев предлагается метод «подстановки многочленов с неопределенными коэффициентами», используя который, в частности, удалось описать алгоритм, позволяющий однозначно ответить на вопрос о локальном минимуме в стационарной точке для полинома, являющегося суммой двух А-квазиоднородных форм, где А – двухмерный вектор, компоненты которого являются натуральными числами..}

%% Ключевые слова на русском
\keywords{полиномы, степенные ряды, необходимые и достаточные условия экстремума, квазиоднородные формы.}

\udk{5XX.XXX}  
\doi{}		 %заполняется редакцией

% \Received{в редакцию \_.\_.202\_, после доработки \_.\_.202\_. Принята к публикации \_.\_.202\_.}
% \received{}
% \revised{}
% \accepted{}
%% дата получения; заполняется редакцией

% %% Поддержка и т.п.
% \thnk{Работа выполнена при финансовой поддержке Российского фонда фундаментальных исследований...}

\section*{Введение}  % Разделы без нумерации
Настоящая работа является продолжением работ \cite{nef:1, nef:2, nef:3}, в которых приводятся некоторые достаточно тонкие необходимые и достаточные условия локального экстремума (для определенности – минимума) в стационарной точке полинома или степенного ряда (а тем самым – аналитической функции). При этом рассматриваются случаи, когда классические методы ~\cite{nef:4}, основанные на использовании матрицы вторых производных «не работают» (в частности, возможны случаи, когда матрица вторых производных нулевая). Приведенные в ~\cite{nef:3} примеры показывают, что полученные в ~\cite{nef:1, nef:2, nef:3} необходимые и достаточные условия близки к тому, чтобы составить общее необходимое и достаточное условие (и тем самым «сомкнуться»). Однако, такое общее условие получено не было, т.е. для исследования некоторых задач требуются еще более тонкие исследования.

Между тем, для случая одной переменной такое условие есть. Если первое ненулевое значение в стационарной точке достигается для производной нечетного порядка, то локального минимума нет, а если для производной четного порядка, то в случае положительного значения имеем локальный минимум, а в случае отрицательного – нет.

Следующим по сложности является случай с двумя переменными. В этом случае многие процедуры, к которым сводится проверка необходимых и достаточных условий, полученных в ~\cite{nef:1, nef:2, nef:3}, основываются на вычислении действительных корней многочлена от одной переменной, а также на решении некоторых других достаточно простых практически реализуемых задач. Следует отметить, что вычисление действительных корней многочлена может быть в настоящее время осуществлено с помощью программ, действующих в онлайн режиме, т.е. не представляет большого труда. При этом удалось описать алгоритм, основанный на этих процедурах, предоставляющий возможность для решения исследуемой задачи даже в очень тонких случаях. Тем не менее все же остаются случаи, когда этот алгоритм «не работает». Для таких случаев предлагается метод «подстановки многочленов с неопределенными коэффициентами», основанный на лемме 7 из ~\cite{nef:2}. Используя этот метод, в частности, удалось описать алгоритм, позволяющий однозначно ответить на вопрос о наличии локального минимума в стационарной точке для полинома, являющегося суммой двух А-квазиоднородных форм, где А – двухмерный вектор, компоненты которого являются натуральными числами. Этот результат имеет с одной стороны самостоятельное значение, а с другой стороны, позволяет применить весьма тонкие достаточные условия локального минимума при использовании теоремы 3.3 из ~\cite{nef:3} для случая двух переменных, в формулировке которой присутствует условие локального минимума для суммы первых нескольких (в частности, двух) квазиоднородных форм.
 
Следует отметить, что поскольку метод «подстановки многочленов с неопределенными коэффициентами» основан на необходимом и достаточном условии локального минимума, то при определенном «искусстве» исследователя отрицательный результат (т.е. контр пример, гарантирующий отсутствие локального минимума) может быть получен всегда. В отличие от этого, обоснование наличия локального минимума для каждой конкретной задачи может потребовать привлечения еще более тонких методов исследования. Тем не менее, автору представляется, что для подавляющего числа задач, возникающих из практики, предлагаемой методики будет достаточно для ответа на исследуемый в работе вопрос.

\section{Постановка задач. Основные сведения} %%\label{ivanov:sec:1}    % Разделы с нумерацией
Будем использовать обозначения и основные понятия из [1-3]. Как и в работе [3] исследуем полином (или, в ряде случаев - степенной ряд) $p(x)$, где $x = (x_1,\dots, x_n) \in \mathbf{R}^n$, в стационарной точке, которая для простоты является нулевой, т.е. в точке $0_{(n)} = (0,\dots,0) \in \mathbf{R}^n$, на наличие локального экстремума (для определённости минимума), т.е. полагаем, что $p(0_{(n)}) = 0$, $p'(0_{(n)}) = 0_{(n)}$.

Нам потребуются некоторые обозначения для вектора $k \in \mathbf{Z}^n_\geq = (\mathbf{Z}_\leq)^n$, где $\mathbf{Z}_\geq = \mathbf{N} \cup \{0\}$, полинома $p(x)$ и множества $N \subseteq \mathbf{Z}^n_\geq : x^k = x_1^{k_1}x_2^{k_2} \cdots x_n^{k_n}$; $\textit{coef}\,(p, k) = a$ -- коэффициент при члене (\emph{мономе}) $ax^k$ полинома $p(x)$; $N = \{k\in \mathbf{Z}^n_\geq \;| \;\textit{coef}\,(p, k) \neq 0\}$ -- носитель полинома $p(x)$; $p_N(x) = \sum\limits_{k \in N}\textit{coef}\,(p, k)x^k$ -- N-\emph{укорочение} полинома $p(x)$. Под \emph{многогранником Ньютона} полинома $p(x) \not\equiv 0$ понимаем множество $\text{Co} N_p$, где $\text{Co Y}$ -- выпуклая оболочка множества $Y \subseteq \mathbf{R}^n$


% \begin{theorem} %% \label{ivanov:thm:1}
%       Теорема. Курсивный шрифт
% \end{theorem}

% \begin{lemma} %% \label{ivanov:lem:1}
%       Лемма. Курсивный шрифт
% \end{lemma}

% \begin{corollary} %% \label{ivanov:cor:1}
%       Следствие. Прямой шрифт
% \end{corollary}

% \begin{proof}
%       Доказательство
% \end{proof}


%%%%%%%%%%%%%     Литература    %%%%%%%%%%%%%

% Источники литературы с кириллическими буквами в описании вводятся командой RBibitem, остальные --- Bibitem. 

% Статьи~\cite{ivanov:1, ivanov:2, ivanov:3, ivanov:16}.

% Книги, диссертации~\cite{ivanov:4,ivanov:5,ivanov:6,ivanov:7,ivanov:8,ivanov:9,ivanov:10,ivanov:11,ivanov:15}.

% Главы, статьи, тезисы в книгах, сборниках~\cite{ivanov:12,ivanov:13,ivanov:14}.

\begin{thebibliography}{99}  %% {9} если используется от 1 до 9 ссылок

% \item[$\bullet$]					Статья в журнале (каждый автор --- на новой строке):
% \vskip0.5cm


\RBibitem{nef:1}	
\by Нефедов В.Н.													  % Фамилия2 И.О.
\paper Необходимые и достаточные условия локального минимума в полиномиальных задачах минимизации. % название статьи
\jour М.: МАИ. 1989. 64с. – Деп. в ВИНИТИ 02.11.89 % название журнала
\issue №6830–В89  													  % том или номер


\RBibitem{nef:2}
\by Нефедов В.Н.
\paper Об оценивании погрешности в выпуклых полиномиальных задачах 
оптимизации.
\jour  ЖВМ и МФ. 1990. 
\vol Т.30
\issue №2
\pages С.200--216


\RBibitem{nef:3}
\by Нефедов В.Н.
\paper Необходимые и достаточные условия экстремума в сложных задачах оптимизации систем, описываемых полиномиальными и аналитическими функциями
оптимизации
\jour Известия РАН. Теория и системы управления
\yr 2023
\issue №2


\RBibitem{nef:4}
\by Васильев Ф.П. 
\book Численные методы решения экстремальных задач
\publaddr М.
\publ Наука
\yr 1980


\Bibitem{nef:5}
\by Viktor Nefedov
\paper Methods and Algorithms for Determining the Main Quasi-homogeneous Forms of Polynomials and Power Series
\jour Matec Web of Conferences 362, 01017
\yr 2022
\URL https://doi.org/10.1051/matecconf/202236201017


\RBibitem{nef:6}
\by Нефедов В.Н. 
\paper Необходимые и достаточные условия экстремума в аналитических задачах оптимизации
\jour Тр. МАИ. Математика.
\yr 2009
\vol №33
\pages 32с.


\RBibitem{nef:7}
\by Гиндикин С.Г. 
\paper Энергетические оценки, связанные с многогранником Ньютона
\jour Тр. Москов. матем. об–ва.
\yr 1974
\vol T.31
\pages C.189-236

\RBibitem{nef:8}
\by Брюно А.Д. 
\book Степенная геометрия в алгебраических и дифференциальных уравнениях.
\publaddr М.
\publ Наука. Физматлит
\yr 1998

\RBibitem{nef:9}
\by Волевич Л.Р.,
Гиндикин С.Г.
\paper Метод многогранника Ньютона в теории дифференциальных уравнений в частных производных
\jour М.: Изд-во Эдиториал УРСС
\yr 2002
\pages 312с.

\RBibitem{nef:10}
\by Хованский А. Г.
\book Многогранники и алгебра  
\publ Тр. ИСА РАН
\yr 2008
\bookvol 38

\RBibitem{nef:11}
\by Нефедов В.Н.
\paper Об одном методе исследования полинома на знакоопределенность в положительном ортанте
\jour  Тр. МАИ. Математика
\yr 2006
\issue №22
\pages 43с.

% \vskip0.5cm\hrule\vskip0.5cm
% \item[$\bullet$]					Статья в архиве (каждый автор --- на новой строке):
% \vskip0.5cm


% \Bibitem{ivanov:3}				
% \by Sidorov P.,
% Kolmogorov B.
% \paper Important properties of an unimportant measure 
% \jour arXiv:3141.1592v1
% \yr 2010


% \vskip0.5cm\hrule\vskip0.5cm
% \item[$\bullet$]					Книга (каждый автор --- на новой строке):
% \vskip0.5cm

% \RBibitem{nef:5}		
% \by Нефёдов В.Н.
% \book Необходимые и достаточные условия локального минимума в полиномиальных задачах минимизации.		  % название книги
% \publaddr Voronezh 											  % место издания
% \publ Vazhnaya kniga										  % издатель
% \yr 2020




% \RBibitem{ivanov:5}
% \by Фейнман Р.,
% Лейтон Р.,
% Сэндс М.
% \book Фейнмановские лекции по физике
% \bookvol 7												      % том книги (если есть)
% \volname Физика сплошных сред								  % название тома (если есть)
% \publaddr М.
% \publ Мир
% \yr 1967

% \RBibitem{ivanov:6}
% \by Фейнман Р.,
% Лейтон Р.,
% Сэндс М.
% \book Физика сплошных сред
% \ser Фейнмановские лекции по физике							  % название серии, в которой выпущена книга или сборник (если есть)
% \sernum 7 													  % номер серии (если есть)
% \publaddr М.
% \publ Мир
% \yr 1967

% \RBibitem{ivanov:7}
% \by Фейнман Р.,
% Лейтон Р.,
% Сэндс М.
% \book Фейнмановские лекции по физике, Т.~7, Физика сплошных сред
% \publaddr М.
% \publ Мир
% \yr 1967



% \Bibitem{ivanov:8}
% \by Feynman R.,
% Leiton P.,
% Sands M.
% \book Physics of solid media
% \edition 4													  % номер издания арабским числом: 1,2,3... (если есть)
% \ser Feynman Lectures in Physics
% \publaddr Moscow
% \publ Mir
% \yr 1967
% \bookvol 7
% \ruslang
% \URL http://kpfu.ru/science/nauchnye-izdaniya/ivrm/pravila


% \RBibitem{ivanov:9}
% \book Современные проблемы геометрии
% \editor Иванова И.И.,									   	  % редакторы указываются, как и авторы, каждый на отдельной строчке,					
% Петрова П.П.												  % в родительном падеже на русском, в именительном - на английском
% \publaddr Москва
% \publ Мир
% \yr 1978


% \vskip0.5cm\hrule\vskip0.5cm
% \item[$\bullet$]					Диссертация (считаем ее книгой):
% \vskip0.5cm

% \RBibitem{ivanov:10}		
% \by Ковалевская М.Ш.
% \book Разложение по собственным функциям на геометрических графах
% \ser дисс. ... канд. физ.-матем. наук
% \publ ВГУ
% \publaddr Воронеж
% \yr 2007
% \lang in Russian											  % то же самое, что \ruslang. Необязательно. Указывается переводчиком, если текст источника не на английском языке. 

% \Bibitem{ivanov:11} % пойдет как book
% \by Kovalevsakya M.Sh.
% \book Decomposition in eigenvalues
% \ser diss. ... cand. phys.-math. sciences
% \publ VGU
% \publaddr Voronezh
% \yr 2007

% \vskip0.5cm\hrule\vskip0.5cm
% \item[$\bullet$]					Статья в сборнике (в книге, в тезисах и т.д., каждый автор --- на новой строке):
% \vskip0.5cm


% \RBibitem{ivanov:12}
% \by Остроградский М.Ю.
% \paper Один подход к нахождению решения нерешенной проблемы
% \inbook Нерешенные проблемы и их решения: тезисы 15й международной конференции, Симферополь
% \publ Наука
% \publaddr Нью-Йорк 
% \pages 132--139
% \yr 2005
% \ruslang % необязательно


% \Bibitem{ivanov:13}
% \by Laplace P.F.,
% Newton I.G.
% \paper Three body motion problem							  % название статьи, тезисов, главы в книге, сборнике, материалах
% \inbook Properties of immobile matter						  % название книги, сборника, материалов
% \ser Mathematical Series
% \sernum 7
% \edition 2
% \bookvol 4
% %\eds St. Ruscheweyh, E.B. Saff, L. C. Salinas, R.S. Varga
% \publ Stinger
% \publaddr Berlin
% \yr 2006
% \URL http://kpfu.ru/science/nauchnye-izdaniya/ivrm/pravila % начинать с http://

% \RBibitem{ivanov:14}
% \by Петров П.П.
% \etal                                                         % команда добавляет 'и др.' или 'et al.' к списку авторов (если необходимо)
% \paper Фракталы на плоскости Лобачевского
% \inbook Современные проблемы геометрии
% \editor Иванова И.И.
% \etal								   	  					  % команда добавляет 'и др.' или 'et al.' к списку редакторов (если необходимо)
% \publaddr Москва
% \publ Мир
% \yr 1978

% \vskip0.5cm\hrule\vskip0.5cm
% \item[$\bullet$]					Если ни один из перечисленных случаев не подходит, источник литературы идет без структуры (нежелательно):
% \vskip0.5cm


% \bibitem{ivanov:15}	
% Бицадзе А.В. \textit{Краевые задачи для эллиптических уравнений второго порядка} (Наука, М., 1966).

% \Bibitem{ivanov:16}
% Саватеев Е.Г. \textit{О задаче идентификации коэффициента параболического уравнения}, Сиб. матем. журн. \textbf{36} (1), 177--185 (1995).

\end{thebibliography}

%% Все остальное оформление - согласно требований на
%% http://kpfu.ru/science/nauchnye-izdaniya/ivrm/pravila


%% Информация об авторе:
% высшее учебное заведение (Академия наук, НИИ, ВЦ и т.д.), \\ служебный адрес (улица, дом, город, индекс, страна),
% e-mail
% например:

% \fullauthor{Имя Отчество Фамилия}
% \address{Казанский  федеральный  университет,\\
% ул. Кремлевская, д. 18, г. Казань, 420008, Россия,}
% \email{***@***.ru}

% % То же для второго автора
% %\fullauthor{ }
% %\address{ }
% %\email{ }

% % для третьего и т.д.

% \vskip0.3truecm
% \begin{center}
% \textsl{Инициалы\,Фамилия на английском}\\[2mm]
% \textbf{Название на английском}
% \end{center}

% %% Данные о статье на английском языке:
% \vskip0.3truecm
% \hbox to \hsize{\hss {\vbox{\hsize=150truemm
% \noindent \small {\sl \small Abstract.} 
% % АННОТАЦИЯ НА АНГЛИЙСКОМ ЯЗЫКЕ, например, 
% In this paper we consider the $A$-integral and its application in the theory of trigonometric series.
% }}\hss} 

% \vskip0.2truecm
% \hbox to \hsize{\hss {\vbox{\hsize=150truemm
% \noindent \small {\sl \small Keywords}:
% \selectlanguage{english} 
% % КЛЮЧЕВЫЕ СЛОВА НА АНГЛИЙСКОМ ЯЗЫКЕ в единственном числе, например,
% $A$-integral, conjugate trigonometric series,
% conjugate function, Cauchy integral, Cauchy type integral.}}\hss}


% %%%% Информация об авторе(ах) на английском языке

% % имя, отчество, фамилия 
% % высшее учебное заведение (Академия наук, НИИ, ВЦ и т.д.), \\ служебный адрес (дом улица, город, индекс страна),
% % e-mail
% % например:

% \fullauthor{Ivan Ivanovich Ivanov}
% \address{Kazan Federal University,\\
% 18 Kremlyovskaya str., Kazan, 420008 Russia,}
% \email{***@***.ru}


\label{lastpage}  %% Не удалять!!!

\end{document}





%%%%%%%%%%%%%%%%%%%%%%%%%%%%%%%%%%%%%%%%%%%%%%%%%%%%

% Примерные рекомендации по подключению графики:
\begin{center}
\begin{figure}[h!]
\includegraphics{___.eps}
% или
% \begin{center} 
%           \includegraphics[scale=0.75]{___.eps}
% масштабирование 75% (если нужно) ^^^^^
%%%%%%                vvvvvvvv
% или \includegraphics[width=...]{___.eps}
% детали см., например, на с.217 в книге "Издательская система LATEX2e"
% И.Котельников, П.Чеботаев, Новосибирск, 1998
\caption{___________}
\label{______}
\end{figure}
\end{center}

% Примерные рекомендации по подключению таблиц:
\begin{center}
\begin{table}[h!]
\begin{tabular}{.........}
%
% Тело таблицы
%
\end{tabular}
\caption{___________}
\label{__________}
\end{table}
\end{center}
