\documentclass[11pt,reqno]{amsart}
%        Обязательные пакеты
\usepackage[T2A]{fontenc}
\usepackage[russian,english]{babel}
\usepackage[utf8]{inputenc}
\usepackage{izvuz7}
\usepackage{amssymb,amsmath,amsthm,amsfonts}
\usepackage{graphicx}

%        Дополнительные пакеты
% \usepackage\bigl[mathscr\bigr]{eucal} % = \mathscr{}, "выпрямленный" рукописный
%        Операторы, например
% \DeclareMathOperator{\const}{const}
% \DeclareMathOperator*{ }{ }  %% оператор с пределами
% \newcommand{\U}{\mathfrak{U}}
% ^^^^^^^^^^^ такие конструкции желательно использовать как можно меньше!


% ВНИМАНИЕ: Из окончательного варианта файла убрать все неиспользуемые окружения!

%        Курсивные окружения
\theoremstyle{plain}
\newtheorem{theorem}{Теорема}
\newtheorem{lemma}{Лемма}
\newtheorem{proposition}{Предложение}
\newtheorem{axiom}{Аксиома}
\newtheorem{conjecture}{Гипотеза}
%% и другие по аналогии при необходимости

%        Подчиненные счетчики в окружениях
%\newtheorem{lemmac}\bigl[theorem\bigr]{Лемма}
%\newtheorem{propositionc}\bigl[theorem\bigr]{Предложение}
%\newtheorem{conjecturec}\bigl[theorem\bigr]{Гипотеза}
% Теорема 1, предложение 2, гипотеза 3...

%        Двойная нумерация окружений
\newtheorem{theorems}{Теорема}[section]
\newtheorem{lemmas}{Лемма}[section]
\newtheorem{propositions}{Предложение}[section]
% Теорема 1.1, лемма 1.1, предложение 1.1 и т.п.

%        Ненумерованные окружения
\newtheorem*{theoremnn}{Теорема}
% Просто теорема (без номера)

%        Окружения с прямым шрифтом
\theoremstyle{definition}
\newtheorem{definition}{Определение}
\newtheorem{corollary}{Следствие}
\newtheorem{remark}{Замечание}
\newtheorem{example}{Пример}

%        Аналогично ненумерованные с прямым шрифтом

% \numberwithin{equation}{section}  %%  для нумерации уравнений типа (1.1)

%        Заголовок статьи.
\tit{\MakeUppercase{НЕОБХОДИМЫЕ И ДОСТАТОЧНЫЕ УСЛОВИЯ ЭКСТРЕМУМА ДЛЯ ПОЛИНОМОВ И СТЕПЕННЫХ РЯДОВ В СЛУЧАЕ ДВУХ ПЕРЕМЕННЫХ}}
% \shorttit{\MakeUppercase{Сокращенный заголовок статьи для нечетных страниц}} % верхний колонтитул на нечетных страницах
 
%        Автор(ы), инициалы, фамилия
\author{\MakeUppercase{В.Н.\,НЕФЁДОВ}}
 

\god{202\_}         %% заполняется редакцией
\nomer{\textnumero\,\_}     %% заполняется редакцией

\setcounter{page}{33}
\pp{\pageref{firstpage}--\pageref{lastpage}}

\begin{document}
\Russian
Покажем теперь, что в случае выполнения условия 4), где $l = 1$, хотя бы при одном $u_0 \in U_{{}^{\textbf{\textasciitilde}}p}(A)$, $0_{(2)}$ не является точкой локального минимума полинома ${}^{\textbf{\textasciitilde}}p (x, y)$ (а тем самым и $p(x, y)$). Для простоты обозначений считаем, что ${{}^{\textbf{\textasciitilde}}}p(x, y) = p(x, y)$.

Таким образом, рассматриваем случай, когда (см. (2)-(4), (8)-(10))%сюда надо вставить ссылки на формулы во время склейки всех частей документа в одну.
\begin{align*}
    &p(x, y) = \varphi_1^A(x,y) + \varphi_2^A(x, y) = x^{\alpha_1}y^{\beta_1}g_1^A(u)+x^{\gamma_1}y^{\eta_1}g_2^A(u)=\\
    &=x^{\alpha_1}y^{\beta_1}(u-u_0)^k\overline{g}_1^A(u) + x^{\chi_1}y^{\eta_1}(u-u_0)\overline{g}_2^A(u), k \in 2\boxtimes
\end{align*}
и при этом, используя (27), (28), имеем:%сюда надо вставить ссылки на формулы во время склейки всех частей документа в одну.
\begin{align*}
    &p(x, y) = (y^{A_1} - u_0x^{A_2})^{kA}\overline{\varphi}_1\;(x, y) + (y^{A_1} - u_0x^{A_2})^A\overline{\varphi}_2\;(x, y),\\
    &\overline{\varphi}_1^A\;(x,y) = x^{\alpha_1-r_1A_2}y^{\beta_1}\Bigl[\, x^{\overline{r_1}A_2}\overline{g}_1^A(x^{-A_2}y^{A_1})\Bigr]\, = x^{\alpha_1-kA_2}y^{\beta_1}\overline{g}^A_1(x^{-A_2}y^{A_1}),\\
    &\overline{\phi_2}^A\;(x,y) = x^{\chi_1-r_2A_2}y^{\eta_1}\Bigl[\, x^{(r_2-1)A_2}\overline{g}_2(x^{-A_2}y^{A_1})\Bigr]\, = x^{\chi_1-A_2}y^{\eta_1}\overline{g}_2(x^{-A_2}y^{A_1}).
\end{align*}

Пусть для некоторых $c_0 \neq 0, d_0 \neq 0$ выполняется $u_0 = c_0^{e_1}d_0^{e_2} = c_0^{-A_2}d_0^{A_1}$. Рассмотрим многочлены:
\[x(t) = c_0t^{A_1}, y(t) = d_0t^{A_2}\pm d_0t^{A_2+\kappa} = d_0t^{A_2}(1 \pm t^\kappa), k \in \boxtimes\]

Тогда (конкретные значения некоторых величин: $h_1,\dots,h_5,\sigma_1,\dots,\sigma_5$ не имеют значения)
\begin{align*}
    &\bigl[\,x(t)\,\bigr]^{-A_2} = \Bigl[\,c_0t^{A_1}\,\Bigr]^{-A_2} = (c_0)^{-A_2}t^{-A_1A_2}, \bigl[\,y(t)\,\bigr]^{A_1} = \Bigl[\,d_0t^{A_2}(1 \pm t^\kappa)\,\Bigr]^{A_1} = (d_0)^{A_1}t^{A_1A_2} + o(t^{A_1A_2}),\\
    &\bigl[\,x(t)\,\bigr]^{-A_2}\bigl[\,y(t)\,\bigr]^{A_1} = (c_0)^{-A_2}(d_0)^{A_1} + O(t) = u_0 + O(t),\\
    &\overline{g}_1^A\Bigl(\,\bigl[\,x(t)\,\bigr]^{-A_2}\bigl[\,y(t)\,\bigr]^{A_1}\Bigr)\, = \overline{g}_1^A(u_0 + O(t)) = \overline{g}_1^A(u_0) + O(t),\\
    &\overline{g}_2^A\Bigl(\,\bigl[\,x(t)\,\bigr]^{-A_2}\bigl[\,y(t)\,\bigr]^{A_1}\Bigr)\, = \overline{g}_2^A(u_0 + O(t)) = \overline{g}_2^A(u_0) + O(t),\\
    &\overline{\varphi}_1^A\bigl(\,x(t), y(t)\bigr)\, = \bigl[\,x(t)\bigr]^{\alpha_1-kA_2}\,\bigl[\,y(t)\bigr]^{\beta_1}\,\overline{g}_1^A\Bigl(\,\bigl[\,x(t)\,\bigr]^{-A_2}\bigl[\,y(t)\,\bigr]^{A_1}\Bigr)\, = h_1t^{\sigma_1} + o\bigl(t^{\sigma_1}\bigr), h_1 > 0, \sigma_1 \in \boxtimes,\\
    &\overline{\varphi}_2^A\bigl(\,x(t), y(t)\bigr)\, = \bigl[\,x(t)\bigr]^{\chi_1-A_2}\,\bigl[\,y(t)\bigr]^{\eta_1}\,\overline{g}_2^A\Bigl(\,\bigl[\,x(t)\,\bigr]^{-A_2}\bigl[\,y(t)\,\bigr]^{A_1}\Bigr)\, = h_2t^{\sigma_2} + o\bigl(t^{\sigma_2}\bigr), h_2 \neq 0, \sigma_2 \in \boxtimes,\\
    &\bigl[\,y(t)\bigr]^{A_1} - u_0\bigl[\,x(t)\bigr]^{A_2}\, = (d_0)^{A_1}\,t^{A_1A_2}\bigl(1 \pm t^\kappa\bigr)^{A_1}\, - u_0(c_0)^{A_2}t^{A_1A_2} =\\
    &= \bigl(d_0\bigr)^{A_1}\,t^{A_1A_2}\, \bigl(\, 1 \pm A_1t^\kappa + o(t^\kappa\bigr)\,)\,-u_0\,(c_0)^{A_2}\,t^{A_1A_2} =\\
    &=\Bigl[\,\bigl(d_0\bigr)^{A_1} - u_0\,(c_0)^{A_2}\Bigr]t^{A_1A_2}\, \pm \bigl(d_0\bigr)^{A_1}\; A_1t^{A_1A_2+\kappa} + o(t^{A_1A_2 + \kappa}) = \pm \bigl(d_0\bigr)^{A_1}\; A_1t^{A_1A_2+\kappa} + o\bigl(t^{A_1A_2+\kappa}\bigr).
\end{align*}
Соответственно,
\[(\,\bigl[\,y(t)\bigr]^{A_1}\, -\,u_0\,\bigl[\,x(t)\bigr]^{A_2}\,)^k\, = h_3t^{\sigma_3+k\kappa} + o(t^{\sigma_3+k\kappa}), h_3 > 0, \sigma_3 \in \boxtimes.\]
Таким образом, получаем
\begin{equation}
\label{nef:eq:1}
\begin{gathered}
    p(x(t),y(t)) = \Bigl(\bigl[\,y(t)\bigr]^{A_1}\, - u_0\,\bigl[\,x(t)\bigr]^{A_2}\,\Bigr)^k \overline{\varphi}_1^A(x(t), y(t)) + \\
    + \Bigl(\bigl[y(t)\bigr]^{A_1} - u_0\,\bigl[x(t)\bigr]^{A_2}\Bigr)\overline{\varphi}_2^A(x(t),y(t)) = h_4t^{\sigma_4+k\kappa} + h_5t^{\sigma_5+\kappa} + o\bigl(t^{\sigma_4+k\kappa}\bigr) + o\bigl(t^{\sigma_5+\kappa}\bigr),
\end{gathered}
\end{equation}
где при соответствующем выборе знака в многочлене $y(t) = d_0t^{A_2}\,\bigl(1\pm t^\kappa\bigr)$ выполняется $h_4 \neq 0, h_5 < 0$, $\sigma_4, \sigma_5 \in \boxtimes$. Выберем число $\kappa \in \boxtimes$ столь большим, чтобы
\[\sigma_4 + k\kappa > \sigma_5 + \kappa.\]
Это можно сделать, поскольку $k \geq 2$ (т.е. достаточно взять любое $\kappa > \sigma_5-\sigma_4$). Тогда из~\eqref{nef:eq:1} получаем
\[p(x(t),y(t)) = h_5t^{\sigma_5+\kappa} + o(t^{\sigma_5+\kappa}), h_5 < 0,\]
т.е. выполняется условие (У3), а следовательно, $0_{(2)}$ не является точкой локального минимума полинома $p(x, y)$.
\section{\textbf{Заключение}}
Используются предложенные в предыдущих работах методы~\cite{nef:1,nef:2,nef:3} для исследования полиномов (или степенных рядов) от двух переменных на наличие локального экстремума в стационарной точке. Предложен легко реализуемый на практике алгоритм и его модификации. Основные шаги этого алгоритма основываются на вычислении действительных корней многочлена от одной переменной, а также на решении некоторых других достаточно простых практически реализуемых задач. Тем не менее, все же остаются случаи, когда этот алгоритм «не работает». Для таких случаев предлагается метод «подстановки многочленов с неопределенными коэффициентами», используя который, в частности, удалось описать алгоритм, позволяющий однозначно ответить на вопрос о локальном минимуме в стационарной точке для полинома, являющегося суммой двух $A$-квазиоднородных форм, где $A$ –- двухмерный вектор, компоненты которого являются натуральными числами.

%%%%%%%%%%%%%     Литература    %%%%%%%%%%%%%

\begin{thebibliography}{99}  %% {9} если используется от 1 до 9 ссылок

\RBibitem{nef:1}	
\by Нефедов В.Н.													  % Фамилия2 И.О.
\paper Необходимые и достаточные условия локального минимума в полиномиальных задачах минимизации. % название статьи
\jour М.: МАИ. 1989. 64с. – Деп. в ВИНИТИ 02.11.89 №6830–В89% название журнала

\RBibitem{nef:2}
\by Нефедов В.Н.
\paper Об оценивании погрешности в выпуклых полиномиальных задачах 
оптимизации.
\jour  ЖВМ и МФ. 1990. Т.30
\issue №2
\pages С.200--216


\RBibitem{nef:3}
\by Нефедов В.Н.
\paper Необходимые и достаточные условия экстремума в сложных задачах оптимизации систем, описываемых полиномиальными и аналитическими функциями
оптимизации
\jour Известия РАН. Теория и системы управления
\yr 2023
\issue №2


\RBibitem{nef:4}
\by Васильев Ф.П. 
\book Численные методы решения экстремальных задач
\publaddr М.
\publ Наука
\yr 1980


\Bibitem{nef:5}
\by Viktor Nefedov
\paper Methods and Algorithms for Determining the Main Quasi-homogeneous Forms of Polynomials and Power Series
\jour Matec Web of Conferences 362, 01017
\yr 2022
\URL https://doi.org/10.1051/matecconf/202236201017


\RBibitem{nef:6}
\by Нефедов В.Н. 
\paper Необходимые и достаточные условия экстремума в аналитических задачах оптимизации
\jour Тр. МАИ. Математика. №33
\yr 2009
\pages 32с.


\RBibitem{nef:7}
\by Гиндикин С.Г. 
\paper Энергетические оценки, связанные с многогранником Ньютона
\jour Тр. Москов. матем. об–ва. T.31
\yr 1974
\pages C.189-236

\RBibitem{nef:8}
\by Брюно А.Д. 
\book Степенная геометрия в алгебраических и дифференциальных уравнениях.
\publaddr М.
\publ Наука. Физматлит
\yr 1998

\RBibitem{nef:9}
\by Волевич Л.Р.,
Гиндикин С.Г.
\paper Метод многогранника Ньютона в теории дифференциальных уравнений в частных производных
\jour М.: Изд-во Эдиториал УРСС
\yr 2002
\pages 312с.

\RBibitem{nef:10}
\by Хованский А. Г.
\book Многогранники и алгебра  
\publ Тр. ИСА РАН
\yr 2008
\bookvol 38

\RBibitem{nef:11}
\by Нефедов В.Н.
\paper Об одном методе исследования полинома на знакоопределенность в положительном ортанте
\jour  Тр. МАИ. Математика
\yr 2006
\issue №22
\pages 43с.

\end{thebibliography}
%% Информация об авторе:
% высшее учебное заведение (Академия наук, НИИ, ВЦ и т.д.), \\ служебный адрес (улица, дом, город, индекс, страна),
% e-mail
% например:

% \fullauthor{Имя Отчество Фамилия}
% \address{Казанский  федеральный  университет,\\
% ул. Кремлевская, д. 18, г. Казань, 420008, Россия,}
% \email{***@***.ru}

% % То же для второго автора
% %\fullauthor{ }
% %\address{ }
% %\email{ }

% % для третьего и т.д.

% \vskip0.3truecm
% \begin{center}
% \textsl{Инициалы\,Фамилия на английском}\\\bigl[2mm\bigr]
% \textbf{Название на английском}
% \end{center}

% %% Данные о статье на английском языке:
% \vskip0.3truecm
% \hbox to \hsize{\hss {\vbox{\hsize=150truemm
% \noindent \small {\sl \small Abstract.} 
% % АННОТАЦИЯ НА АНГЛИЙСКОМ ЯЗЫКЕ, например, 
% In this paper we consider the $A$-integral and its application in the theory of trigonometric series.
% }}\hss} 

% \vskip0.2truecm
% \hbox to \hsize{\hss {\vbox{\hsize=150truemm
% \noindent \small {\sl \small Keywords}:
% \selectlanguage{english} 
% % КЛЮЧЕВЫЕ СЛОВА НА АНГЛИЙСКОМ ЯЗЫКЕ в единственном числе, например,
% $A$-integral, conjugate trigonometric series,
% conjugate function, Cauchy integral, Cauchy type integral.}}\hss}


% %%%% Информация об авторе(ах) на английском языке

% % имя, отчество, фамилия 
% % высшее учебное заведение (Академия наук, НИИ, ВЦ и т.д.), \\ служебный адрес (дом улица, город, индекс страна),
% % e-mail
% % например:

% \fullauthor{Ivan Ivanovich Ivanov}
% \address{Kazan Federal University,\\
% 18 Kremlyovskaya str., Kazan, 420008 Russia,}
% \email{***@***.ru}


\label{lastpage}  %% Не удалять!!!

\end{document}





%%%%%%%%%%%%%%%%%%%%%%%%%%%%%%%%%%%%%%%%%%%%%%%%%%%%

% Примерные рекомендации по подключению графики:
\begin{center}
\begin{figure}\bigl[h!\bigr]
\includegraphics{___.eps}
% или
% \begin{center} 
%           \includegraphics\bigl[scale=0.75\bigr]{___.eps}
% масштабирование 75% (если нужно) ^^^^^
%%%%%%                vvvvvvvv
% или \includegraphics\bigl[width=...\bigr]{___.eps}
% детали см., например, на с.217 в книге "Издательская система LATEX2e"
% И.Котельников, П.Чеботаев, Новосибирск, 1998
\caption{___________}
\label{______}
\end{figure}
\end{center}

% Примерные рекомендации по подключению таблиц:
\begin{center}
\begin{table}\bigl[h!\bigr]
\begin{tabular}{.........}
%
% Тело таблицы
%
\end{tabular}
\caption{___________}
\label{__________}
\end{table}
\end{center}
