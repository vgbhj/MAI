\numberwithin{equation}{section}  %%  для нумерации уравнений типа (1.1)
\renewcommand*{\proof}{\textbf{Доказательство.}}
\newtheorem{statement}{Утверждение}




\begin{remark} {Следует отметить, что в утверждениях теоремы 1 и леммы 1 рассматриваются A-квазиоднородные формы полинома $p(x)$  не для любого вектора A $\in \mathbf{Z}^n\setminus {0_{(n)}}$, а лишь A $\in \mathbf{N}^n$ (в двухмерном случае это соответствует множеству граней «юго-западной» части многогранника Ньютона). Между тем во многих других задачах могут понадобиться  A-квазиоднородные формы полинома $p(x)$ для всех векторов A $\in \mathbf{Z}^n\setminus {0_{(n)}}$,  например, при исследовании полинома на знакоопределенность в положительном ортанте (см.~\cite{nef:11}, а также приведенные там ссылки на предыдущие работы автора).}
\end{remark}
\begin{remark} {В [2-5] описаны практически реализуемые алгоритмы выделения всех форм полинома $p(x)$, являющихся его главными  A-квазиоднородными формами хотя бы при одном A $\in \mathbf{N}^n$, что облегчает возможность применения теоремы 1 и леммы 1. Там же приведены некоторые модификации этих алгоритмов для случая, когда ищутся все формы полинома $p(x)$, представляющие собой его главные  A-квазиоднородные формы при всех возможных A $\in  \mathbf{Z}^n\setminus {0_{(n)}}$.}
\end{remark}
\begin{remark}{В случае A $\in \mathbf{N}^n$ можно ставить задачу о нахождении совокупности главных  A-квазиоднородных форм не только для полинома, но и для степенного ряда, который будет результатом разложения аналитической функции по степеням переменных в окрестности стационарной точки, поскольку в этом случае их число конечно и потребуется конечное число «первых» членов из этого разложения. При этом теорема 1 и лемма 1 остаются в силе и для степенного ряда $p(x)$~\cite{nef:6}.}
\end{remark}

Как видно из простых примеров (см. пример 6 из~\cite{nef:3}), использование только теоремы 1 и леммы 1, не всегда приводит к решению исследуемой задачи. Существенным продвижением является использование помимо главных форм полинома его разложения на сумму квазиоднородных форм. Для любого полинома $p(x)\not\equiv 0$, для любого A $\in \mathbf{Z}^n\setminus {0_{(n)}}$ можно разложить этот полином на сумму A-квазиоднородных форм вида \begin{equation}\label{nef:eq:1} {p(x) = \phi_1 +\dots+\phi_r,} \end{equation} где $\phi_i\not\equiv0$; $\forall k \in N_{\phi_i}$ $\langle A,k\rangle = B_{i}\in\mathbf{Z}$,$i=\overline{1,r}$, $r \in \mathbf{N}$, и при этом $B_1 < B_2 < \dots < B_r$. Если, кроме того, $A \in \mathbf{N}^n$, $p(0_{(n)})= 0$, то $B_1 \in \mathbf{N}$. Обозначим для любого полинома $p(x)$ $H_p = \{x \in \mathbf{R}^n|p(x) = 0\} $

Для многочлена $q(t)=b_{1}t^{k_1}+b_{2}t^{k_2}+\dots+b_{l}t^{k_l}$, где $b_1,\dots,b_l \in\mathbf{R}\setminus\{0\}$, $k_1,\dots,k_l \in \mathbf{N}$, $1\leq k_1 < \dots < k_l$, зависящего от переменной $t \in \mathbf{R}$, кратко запишем (выделяя главный член  в этом многочлене) $q(t)=b_{1}t^{k_1}+o(t^{k_1})$ или $q(t)=b_{1}t^{k_1}+O(t^{k_1+1})$ (также $q(t)=b_{1}t^{k_1}+O(t^{k_2})$ - в случае $l\geq 2 $). Аналогичные записи можно использовать и в случае $l=1$, а также, если требуется выделить в $q(t)$ первые несколько главных членов. Используем следующие простые факты относительно многочленов.

Пусть $k \in \mathbf{Z}^{n}_{\geq} \setminus\{0_{(n)}\}$, $p(x)=x^k$ - моном, $A=(A_1,\dots,A_n)\in\mathbf{N}^n$, $\langle A,k\rangle = B\in\mathbf{N}$, $a_i\neq 0$, $q_i(t)=a_it^{A_i} +o(t^{A_i})$ – многочлены от переменной $t\in\mathbf{R}$, $i=\overline{1,n}$, $a=(a_1,\dots,a_n)\in\mathbf{R}^n$. Тогда $p(q_1(t),\dots,q_n(t))=a^kt^B+o(t^B)$.

Пусть $A=(A_1,\dots,A_n) \in \mathbf{N}^n,\phi(x)\not\equiv 0$ – полином, являющийся A-квазиоднородной полиномиальной формой, $\forall k \in\mathbf{N}_\phi  \langle A,k\rangle = B\in\mathbf{N}$, $a=(a_1,\dots,a_n)\in\mathbf{R}^n$ $i=\overline{1,n}$. Тогда $\phi(a_1t^{A_1},\dots,a_nt^{A_n})=\phi(a)t^B$. Если $q_i(t)=a_it^{A_i}+o(t^{A_i})$ – многочлены от переменной  $t\in\mathbf{R}$, $a_i\neq0$, $i=\overline{1,n}$, $\phi(q_1(t),\dots,q_n(t))=\phi(a)t^B+o(t^B)$.

Лемму 1, а также теорему 1 можно усилить.

\begin{lemma}~\cite{nef:3} \label{nef:lem:2} {Пусть $0_{(n)}$  – точка локального минимума полинома $p(x)$, $p(0_{(n)})=0$, $p^\prime(0_{(n)})=0_{(n)}$. Тогда $\forall A \in\mathbf{N}^n$, для разложения (1) выполняется $\forall i \in\{1,\dots,r\}$, $\forall x\in H_{\phi_1}\cap\dots\cap H_{\phi_{i-1}}$ $\phi_i(x)\geq 0$ (в частности, при $i=1$ имеем $\forall x \in\mathbf{R}^n$ $\phi_1(x)\geq0$).	}
\end{lemma}

\begin{theorem}~\cite{nef:3} \label{nef:thm:2} {Пусть $p(x)$ – полином, $p(0_{(n)})=0$, $p^\prime(0_{(n)})=0_{(n)}$, $\forall A \in\mathbf{N}^n$ все главные A-квазиоднородные формы $p(x)$ неотрицательны (т.е. выполняются необходимые условия минимума из леммы 1) . Пусть для любого $A\in\mathbf{N}^n$ для разложения (1) справедливо, что для некоторого $j \in \{2,\dots,r\}$ $0_{(n)}$ – точка локального минимума полиномов $\phi_1(x)$,  $\phi_1(x)+\phi_2(x)$,$\ldots$,$\phi_1(x)+\dots+\phi_{j-1}(x)$. При этом либо $H_{\phi_1}\cap\dots\cap H_{\phi_{j-1}}\cap\bigl[R\setminus\{0\}\bigr]^n =\varnothing$,либо $\forall x\in H_{\phi_1}\cap\dots\cap H_{\phi_{j-1}}\cap\bigl[R\setminus\{0\}\bigr]^n$ $\phi_j>0$. Тогда $0_{(n)}$ – точка локального минимума $p(x)$.}
\end{theorem}

\begin{remark}{В случае $n=2$ далее в разделе 3 будет описан практически реализуемый алгоритм, позволяющий ответить на вопрос, является ли $0_{(2)}$  точкой локального минимума полиномов $\phi_1(x)$, $\phi_1(x)+\phi_2(x)$.	}
\end{remark}

\section{Случай двух переменных} \label{nef:sec:2}
В этом разделе рассматриваются полиномы (степенные ряды) от двух переменных, т.е.  случай $n=2$,$x=(x_1,x_2)\in \mathbf{R}$. Для простоты обозначений всюду далее полагаем $x=x_1$, $y=x_2$, $p(x_1,x_2)=p(x,y)$.Пусть $p(x,y)\not\equiv0$, $p(0,0)=0$, $p^\prime(0,0)=0_{(2)}$, т.е. $0_{(2)}$ является стационарной точкой. Будем проводить исследования, является ли $0_{(2)}$ точкой локального минимума  $p(x,y)$.

Рассмотрим сначала простой случай, когда $p(x,y)$ является A-квазиоднородной формой  для некоторого вектора $\mathbf{m}A=(A_1,A_2)\in\mathbf{N}^2$, т.е. имеет вид $p(x,y)=\phi^A_{1}(x,y)$, где для некоторого $s\in\mathbf{N}$ выполняется \begin{equation}\label{nef:eq:2}
	\phi^{A}_{1}(x,y)=\sum_{i=1}^{s}a_{i}x^{a_{i}}y^{b_{i}}
\end{equation}
\begin{equation}\label{nef:eq:3} 
	a_i\neq0,\,  \alpha_i,\beta_i\in\mathbf{N}\cup \{0\},\, A_i\alpha_i+A_2\beta_i=B_1\in\mathbf{N},\,i=1,\dots,s,\,   \alpha_1>\alpha_2>\ldots>\alpha_s\geq0
\end{equation}
Будем для простоты считать, что $\text{НОД}(A_1,A_2=1)$ (иначе разделим $A_1$,$A_2$ на $\text{НОД}(A_1,A_2)$). Точки $(\alpha_i,\beta_i)$ находятся на одной прямой, координаты $(\alpha,\beta)$  которой удовлетворяют уравнению $A_1\alpha+A_2\beta=B_1$ или $\frac{\alpha-\alpha_1}{-A_2}=\frac{\beta-\beta_1}{A_1}$, с направляющим вектором $e=(-A_2,A_1)$, а следовательно, $(\alpha_i,\beta_i)=(\alpha_1,\beta_1)+\nu_ie$, $i=1,\ldots,s$, где $\nu_1=0$, $\nu_i>0$, $i=2,\ldots,s$. Используя то, что $\text{НОД}(|e_1|,|e_2|)=\text{НОД}(|A_1|,|A_2|)=1$,$(\alpha_i-\alpha_1,\beta_i-\beta_1)=\nu_ie$, получаем, что $\nu_i\in\mathbf{N}$,$i=2,\ldots,s$ $(\nu_1=0)$.Таким образом, при $x\not=0$ имеем:
\begin{equation}\label{nef:eq:4} 
	\begin{aligned}
		&\phi^A_1(x,y)=\sum_{i=1}^{s}a_ix^{\alpha_i}y^{\beta_i}=\sum_{i=1}^{s}a_ix^{\alpha_1+\nu_ie_1}y^{\beta_1+\nu_ie_2}=x^{\alpha_1}y^{\beta_1}\sum_{i=1}^{s}a_i\big(x^{e_1}y^{e_2}\big)^{\nu_i}=\\
		&=x^{\alpha_1}y^{\beta_1}\sum_{i=1}^{s}a_iu^{\nu_i}=x^{\alpha_1}y^{\beta_1}g_1(u),\,u=x^{e_1}y^{e_2},\,e=(-A_2,A_1),\, g^A_1(u)=\sum_{i=1}^{s}a_iu^{\nu_i}
	\end{aligned}
\end{equation}
где $g^A_1(u)$ - многочлен от одной переменной  $u\in\mathbf{R}$ (при этом в силу $\text{НОД}(|e_1|,|e_2|)=1$ хотя бы одно из чисел среди $e_1$,$e_2$  является нечетным, а следовательно, переменная $u=x^{e_1}y^{e_2}$  может принимать при $(x,y)\in\mathbf{R}^2$  любые действительные значения). Будем в дальнейшем $g^A_1(u)$ называть \emph{характеристическим многочленом} для квазиоднородной формы $\phi^A_1(x,y)$, а $a_ix^{\alpha_i}y^{\beta_i}$ – \emph{главным мономом}. Соответственно, в случае $x=0$ имеем: $p(0,y)\equiv0$ при $\alpha_s>0$,  и $p(0,y)=\alpha_sy^{\beta_s}$ при $\alpha_s=0$.

Справедливо следующее простое
\begin{statement} \label{nef:stat:1} Полином $p(x,y)=\phi^A_1(x,y)$ вида~\eqref{nef:eq:2},~\eqref{nef:eq:3}, где $\mathbf{m}A=(A_1,A_2)\in\mathbf{N}^2$, является неотрицательным тогда и только тогда, когда выполняются условия:\linebreak
	\indent $1)$ $a_1>0$, $a_s>0$,$\alpha_1,\beta_1,\alpha_s\beta_s\in 2\big(\mathbf{N}\cup\{0\}\big)$;\newline
	\indent$2)$ многочлен $g^A_1(u)=\sum_{i=1}^{s} a_iu^{\nu_i}$ является неотрицательным, т.е. (в силу того, что $g^A_1(0)=a_1>0$) либо не имеет действительных корней, либо все его действительные корни имеют четную кратность.
\end{statement}
\noindent \begin{proof} 
	Докажем необходимость (достаточность очевидна). Покажем сначала справедливость 1). Например, если $a_1>0$, $\alpha_1$ – четно, $\beta_1$ – нечетно, то $p(t^{A_1},-t^{A_2+1})=-a_1t^{B_1+\beta_1}=o(t^{B_1+\beta_1})$, т.е $p(x,y)$ е является неотрицательным. Остальные случаи невыполнения условия 1) рассматриваются аналогично. Покажем теперь справедливость 2). Предположим, что для некоторого $u_0\in\mathbf{R}$ выполняется $g^A_1(u_0)<0$. Тогда $u_0\not=0$ (поскольку $g^A_1(0)=a_1>0$).Используя то, что хотя бы одно из чисел среди $e_1$,$e_2$ является нечетным, нетрудно подобрать $x\not=0$,$y\not=0$,являющиеся решениями уравнения $u_0=x^{e_1}y^{e_2}$ (например, если $e_1$ – нечетно, то полагаем: $y=1$, $x=(u_0)^{1/e_1}$).Тогда в силу~\eqref{nef:eq:4} при выбранных $x$, $y$  выполняется (с учетом четности $\alpha_1$,$\beta_1$) $p(x,y)=x^{\alpha_1}y^{\beta_1}g^A_1(u_0)<0$. 
\end{proof}
\begin{remark}
	Следует также отметить, что если полином $p(x,y)=\phi^A_1(x,y)$ вида~\eqref{nef:eq:2},~\eqref{nef:eq:3} является квазиоднородной формой для некоторого вектора $\mathbf{m}A=(A_1,A_2)\in\mathbf{Z}^2\setminus\{0_{(2)}\}$, такого, что $\mathbf{m}A_1\geq0$, $\mathbf{m}A_2\leq0$, $a_1>0$, $\alpha_1,\beta_1\in2\big(\mathbf{N}\cup\{0\}\big)$ (при этом в случае $\mathbf{m}A_2<0$ заменяем в~\eqref{nef:eq:3} условие $\alpha_1>\alpha_2>\ldots>\alpha_s$ на $\alpha_1<\alpha_2<\ldots<\alpha_s$, а в случае $\mathbf{m}A_2=0$, $\mathbf{m}A_1>0$  на условие $\beta_1<\beta_2<\ldots<\beta_s$ ), $0_{(0)}$  – точка локального минимума этого полинома. Действительно, в любом из этих случаев одночлен (моном) $a_1x^{\alpha_{1}}y^{\beta_1}$ является главным членом во всей квазиоднородной форме $p(x,y)\equiv\phi^A_1(x,y)$,  поскольку любой другой член этого полинома имеет вектор степеней превосходящий по Парето вектор $(\alpha_1,\beta_1)$.  Соответственно, в случае невыполнения хотя бы одного из условий $a_1>0$, $\alpha_1,\beta_1\in2\big(\mathbf{N}\cup\{0\}\big)$ $0_{(2)}$ – не является точкой локального минимума рассматриваемого полинома. Заметим при этом, что в утверждении 1 и замечании 5 рассмотрены A-квазиоднородные формы для всех $\mathbf{m}A=(A_1,A_2)\in\mathbf{Z}^2\setminus\{0_{(2)}\}$,  поскольку A-квазиоднородная форма является одновременно -квазиоднордной формой для всех $\lambda \in\mathbf{Z}\setminus\{0\}$  (в то числе для $\lambda=-1$).
\end{remark}	
Таким образом, в соответствии с утверждением 1 в случае, если полином $p(x,y)=\phi^A_1(x,y)$, где $\mathbf{m}A\in\mathbf{N}^2$, имеет вид~\eqref{nef:eq:2},~\eqref{nef:eq:3} для исследования его на неотрицательность следует сначала проверить выполнение условий: $a_1>0$,$a_s>0$, $\alpha_1,\beta_1,\alpha_s,\beta_s\in2\big(\mathbf{N}\cup\{0\}\big)$. Если хотя бы одно из этих условий не выполняется, то этот полином не является неотрицательным. Если же эти условия выполняются, то проверяем выполнение условия 2) утверждения 1. Для этого во-первых, находим действительные корни  многочлена $g^A_1(u)$ из~\eqref{nef:eq:3}. При этом возможны случаи:\newline
\indent (a) многочлен $g^A_1(u)$  не имеет действительных корней и тогда, поскольку $g^A_1(0)=a_1>0$, выполняется: $g^A_1(u)>0$ при всех $u\in\mathbf{R}$ , а следовательно, в силу утверждения 1 полином $p(x,y)$ является неотрицательным;\newline
\indent (б) многочлен $g^A_1(u)$ имеет действительные корни. В этом случае очевидным условием неотрицательности $g^A_1(u)$
вляется четная кратность его корней.

Проведем теперь исследование, является ли $0_{(2)}$ точкой локального минимума полинома $p(x,y)$, для общего случая. Поскольку полином $p(x,y)$ зависит от двух переменных, то многогранник Ньютона $\text{Co}N_p$ этого полинома может иметь размерность 1 или 2. Если он имеет размерность 1, то этот полином удовлетворяет условиям либо утверждения 1, либо замечания 5. В любом из этих случаев полином $p(x,y)$ легко исследуется на наличие в точке $0_{(2)}$  –  локального минимума этого полинома. При этом, если выполняются условия утверждения 1, то $0_{(2)}$ является одновременно точкой локального и глобального минимума этого полинома на $\mathbf{R}^2$.

Рассмотрим теперь более сложный случай, когда $\text{Co}N_p$  имеет размерность 2. В этом случае его главные А-квазиоднородные формы для произвольного $\mathbf{m}A=(A_1,A_2)\in\mathbf{N}^2$ можно разделить на 3 группы. Будем при этом предполагать, что $\text{НОД}(A_1,A_2)=1$ (поскольку A-квазиоднородная форма является одновременно $\lambda A$ – квазиоднородной формой для всех $\lambda\in\mathbf{Q}\setminus\{0\}$ таких, что $\lambda A\in\mathbf{Z}^2$, где $\mathbf{Q}$ – множество рациональных чисел).
\newline Будем обозначать $\mathbf{N}^2_0=\big\{A=(A_1,A_2)\in\mathbf{N}^2\big|\,\text{НОД}(A_1,A_2)=1 \big\}$.

\emph{Группа 1.} В эту группу входят главные квазиоднородные формы полинома $p(x,y)$, соответствующие граням многогранника $\text{Co}N_p$ размерности 0. Каждая из таких форм представляет собой моном вида $ax^{\alpha}y^{\beta}$, являющийся членом полинома $p(x,y)$. Как отмечалось в~\cite{nef:3}, эти мономы соответствуют угловым точкам многогранника $\text{Co}N_p$, т.е. $(\alpha,\beta)\in\Psi(N_p)\subseteq N_p$, где через $\Psi(N_p)$ обозначается в~\cite{nef:3} множество угловых точек $\text{Co}N_p$. Более того, векторы $(\alpha,\beta)$ являются оптимальными по Парето на множестве $N_p$ , т.е. $(\alpha,\beta)\in P(N_p)$, где $P(N_p)$ - множество оптимальных по Парето точек из $N_p$. Таким образом, $(\alpha,\beta)\in\Omega(N_p)$,  где в соответствии с обозначениями из~\cite{nef:3} $\Omega(N_p)=P(N_p)\cap\Psi(N_p)$. Их выделение (даже визуальное, с использованием графического изображения многогранника $\text{Co}N_p$)
вляется достаточно простой задачей (см. в [1, 5] алгоритмы их выделения для произвольного $\emph{n}$). Необходимым и достаточным условием неотрицательности таких форм, которая в этом случае влечет также невырожденность в слабом смысле, является (см. лемму 2 из~\cite{nef:3})
\begin{equation} \label{nef:eq:5}
	a>0,\,\alpha,\beta\in2\big(\mathbf{N}\cup\{0\}\big)
\end{equation}

Будем в дальнейшем считать, что это условие выполнено для всех мономов из группы 1 (в первом же случае его невыполнения рассматриваемая задача решена и $0_{(2)}$ не является точкой локального минимума полинома $p(x,y)$ ).

\emph{Группа 2.} В эту группу войдут главные квазиоднородные формы полинома $p(x,y)$ размерности 1 (т.е. соответствующие сторонам многогранника $\text{Co}N_p$), представляющие собой сумму двух мономов, каждый из которых в этом случае соответствует одной из двух угловых точек этой стороны, являющихся угловыми точками $\text{Co}N_p$ и одновременно принадлежащих $P(N_p)$. Таким образом, в группу 2 войдут некоторые суммы квазиоднородных форм из группы 1. Поскольку для мономов, входящих в эту сумму, предполагается выполнение условия вида~\eqref{nef:eq:5}, обеспечивающее неотрицательность и невырожденность в слабом смысле каждого из них, то и сумма этих мономов будет, очевидно, также неотрицательной и невырожденной в слабом смысле.

\emph{Группа 3.} В эту группу войдут все остальные главные квазиоднородные формы полинома $p(x,y)$ (не вошедшие в группы 1, 2), соответствующие граням (сторонам) многогранника $\text{Co}N_p$ ,  размерности 1, являющиеся суммами по крайней мере трех мономов (два соответствуют угловым точкам этой грани и имеется по крайней мере один моном, соответствующий промежуточной точке этой грани, находящейся между ее угловыми точками). Все дальнейшие исследования будут посвящены исследованию случаев, когда главная квазиоднородная форма принадлежит группе 3. 

Пусть мы выделили одну из главных квазиоднородных форм полинома $p(x,y)$, принадлежащих группе 3. Тогда для некоторого $\mathbf{m}A=(A_1,A_2)\in\mathbf{N}^2_0$ она имеет вид~\eqref{nef:eq:2},~\eqref{nef:eq:3}, а сам полином $p(x,y)$ имеет вид
\begin{equation} \label{nef:eq:6}
	p(x,y)=\phi^A_1(x,y)+\phi^A_2(x,y)+\ldots+\phi^A_{r_A},
\end{equation}
\begin{equation}
	\phi^A_i(x,y)\not\equiv0;\, \forall k\in\mathbf{N_{\phi_i}}\,\langle A,k\rangle=B_i\in\mathbf{N},\, i=1,2,\ldots,r_A,\, B_1<B_2<\ldots<B_{r_A},
\end{equation}
где $r_A\geq2$ (поскольку рассматривается случай  $\text{dim} \, \text{Co}N_p=2$~\cite{nef:4}).

Задача определения вектора $\mathbf{m}A=(A_1,A_2)\in\mathbf{N}^2_0$ по членам выделенной квазиоднородной формы (например, они могут быть выделены визуально, исходя из изображения $\text{Co}N_p$; см. также в [1, 5] алгоритмы их выделения для произвольного $\emph{n}$) является простой вычислительной задачей. Этот вектор определяется по любым двум членам, входящим в эту форму, например, $a_1x^{\alpha_1}y^{\beta_1}$, $a_2x^{\alpha_2}y^{\beta_2}$